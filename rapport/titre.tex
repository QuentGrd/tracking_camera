\begin{titlepage}{
    \begin{center}
        \vspace* {35mm}
        {\Large \textbf {Université de Cergy-Pontoise}} \\
        Cours d'informatique embarqué et robotique - Pierre Andry\\
       
        \vspace* {10mm}


        \vspace* {10mm}
	{\Huge \textsf{Tracking couleur}} \\
		\vspace* {10mm}
        \textbf {Vilain Matthieu et Quentin Gerard} \\
        L3 CMI SIC \\
        \vspace* {20mm}
 	
	\end{center}
	
	\begin{abstract}
		Cette expérience à était réalisé dans le cadre du cours d'informatique embarquée et robotique de l'université de Cergy-Pontoise enseigné par Pierre Andry. Le but de cette expérience est de réaliser un dispositif de tracking "pan-tilt" suivant une couleur préférentielle.\\
    	Dans un premier temps nous allons voir l'efficacité d'une méthode d'asservissement sur une cible fixe puis essayer sur cible mouvante et enfin essayer une autre méthode d'asservissement.\\
    	Enfin nous verrons comment nous avons essayé d'améliorer la détection de la couleur afin d'améliorer le tracking.
	\end{abstract}
}
\end{titlepage}
